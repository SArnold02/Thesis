\chapter{Application testing}
\label{chap:testing}

\section{Testing techniques}
\label{sec:testingsec1}

\par For testing the application the most used technique was exploratory testing. This method can also be well described by trial and error, since its building blocks are trying out different scenarios of the application, analyzing the result, learn why things succeeded or failed and finally design the application in future with new found knowledge from the previous step. In the case of the application the testing was done by hand and by one person.

\section{Use case testing specifics}
\label{sec:testingsec2}

\par In the case of the F1 functionality, starting the recording, the main focus was that the frames are correctly shown and saved. To test this, a thorough check was made to ensure the route of the current frame by using debug tools. The saving was checked during recording by copying the file and also after it finished in order to check that the images are saved correctly. To check the audio the route was checked and the final output after the capture was done. The final step was to ensure that the correct folder was used as the saving destination by changing the path multiple times between recordings, and checking the video files as mentioned before.